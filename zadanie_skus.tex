\documentclass{article}
\usepackage{framed}
\usepackage{fancyvrb}
\usepackage{color}
\usepackage{hyperref}
\usepackage{amsmath}
\newcommand{\codefile}[1]{\begin{framed}\input{#1}\end{framed}}
\newcommand{\codefilesmall}[1]{\begin{framed}\small\input{#1}\end{framed}}
\newcommand{\codefilenoframedsmall}[1]{{\small\input{#1}}}
\newcommand{\codefilenoframedtiny}[1]{{\tiny\input{#1}}}
\newcommand{\tvs}{\textvisiblespace}
\input{pygments}
\begin{document}
\title{Transformácia jednoduchého značkovacieho jazyka do HTML\\
(skúšobné zadanie OS ZS 2021/22)}
\maketitle

\section{Zadanie}

Rozšírte {\tt markdown.sh} z domáceho zadania tak, aby podporoval tieto
ďalšie syntaktické konštrukty.

\subsection{URL}
Kdekoľvek na riadku sa nachádza reťazec typu {\tt <https://}{\em text}{\tt>},
nahraďte ho podľa pravidla
\begin{center}
{\tt <https://}{\em text}{\tt>}
$\quad\to\quad$
{\tt <a href="https://}{\em text}{\tt ">https://}{\em text}{\tt </a>}
\end{center}
napríklad riadok
\codefile{url.md}
sa transformuje na
\codefile{url.html}

\subsection{Zoznamy}

Začnite s tým, že zmeníte 
\codefile{whileold.sh}
na
\codefile{whilenew.sh}
Je to kvôli tomu, že príkaz {\tt read LINE} za normálnych okolností
zahadzuje počiatočné medzery na riadku.

Transformujte každú sekvenciu riadkov typu
\codefile{list.md}
na sekvenciu typu
\codefile{list.html}
To znamená
\begin{itemize}
\item Ak narazíte na prvý výskyt {\tt \tvs-\tvs}\footnote{\tvs ~znamená medzeru.} na začiatku riadku vypíšte najprv
riadok {\tt <ul>} 
\item
Riadky v zozname transformujte podľa pravidla
\begin{center}
{\tt\tvs-\tvs}{\em text}$\quad\to\quad${\tt <li>}{\em text}{\tt </li>} 
\end{center}
\item Ak narazíte na nejaký riadok ktorý nezačína {\tt \tvs-\tvs} a
\underline{predtým bol zoznam}, je to koniec zoznamu a pred ďalším spracovaním
vypíšte {\tt </ol>}, aby ste ten zoznam správne ukončili.
\end{itemize}

\section{Pomôcky}

\subsection{Použite spätné referencie}

V prvej časti je rozumné použiť {\tt sed /}{\it regex}/{\it replacement}{\tt /}, pričom v {\it replacement} budete referencovať cez {\tt $\backslash$1} časť {\it regex}, ktorú ste vyznačili pomocou $\backslash(\dots\backslash)$.

\subsection{Pomocná premenná}
Pre spracovanie zoznamov potrebujete nejakú pomocnú premennú, kde si budete
pamätať, či ste uprostred zoznamu alebo nie, povedzme {\tt ZOZ}. Jej rovnosť
nejakej hodnote môžete zistiť pomocou {\tt if} a {\tt test}
\codefile{iftest.sh}
prípadne (pozor na medzery!)
\codefile{ifbra.sh}
\newpage
\subsection{Vstup (príklad)}
\codefilenoframedsmall{example.md}
\newpage
\subsection{Výstup (príklad)}
\codefilenoframedtiny{example.html}
\end{document}
