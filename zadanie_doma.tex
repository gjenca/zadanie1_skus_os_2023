\documentclass{article}
\usepackage{framed}
\usepackage{fancyvrb}
\usepackage{color}
\usepackage{hyperref}
\usepackage{amsmath}
\newcommand{\codefile}[1]{\begin{framed}\input{#1}\end{framed}}
\newcommand{\codefilesmall}[1]{\begin{framed}\small\input{#1}\end{framed}}
\input{pygments}
\begin{document}
\title{Transformácia jednoduchého značkovacieho jazyka do HTML\\
(domáce zadanie OS ZS 2021/22)}
\maketitle

\section{Zadanie}
Vytvorte skript {\tt markdown.sh}, ktorý bude na štandardnom vstupe
očakávať vstup v značkovacom jazyku, ktorý je čiastočnou implementáciou
\href{https://www.markdownguide.org/}{jazyka markdown}. Na výstupe bude produkovať
HTML podľa doleuvedených pokynov.

Použitie bude teda napríklad
\begin{framed}
\begin{Verbatim}[commandchars=\\\{\}]
./markdown.sh \PYZlt{} input.md \PYZgt{} output.html
\end{Verbatim}

\end{framed}
a výstup potom bude v jazyku HTML. Výstup si potom môžete otvoriť v nejakom internetovom
prehliadači, napríklad
\begin{framed}
\begin{Verbatim}[commandchars=\\\{\}]
firefox output.html
\end{Verbatim}

\end{framed}
\section{Notácia}
V doleuvedenom značí znak {\textvisiblespace} medzeru.
\section{Požadovaný výstup}
\subsection{Hlavička}
Na začiatku skript vypíše konštantný text
\begin{framed}
\small
\begin{Verbatim}[commandchars=\\\{\}]
\PY{c+cp}{\PYZlt{}!DOCTYPE html\PYZgt{}}
\PY{p}{\PYZlt{}}\PY{n+nt}{html}\PY{p}{\PYZgt{}}
\PY{p}{\PYZlt{}}\PY{n+nt}{head}\PY{p}{\PYZgt{}}
  \PY{p}{\PYZlt{}}\PY{n+nt}{meta} \PY{n+na}{http\PYZhy{}equiv}\PY{o}{=}\PY{l+s}{\PYZdq{}Content\PYZhy{}type\PYZdq{}} \PY{n+na}{content}\PY{o}{=}\PY{l+s}{\PYZdq{}text/html;charset=UTF\PYZhy{}8\PYZdq{}} \PY{p}{/}\PY{p}{\PYZgt{}}
\PY{p}{\PYZlt{}}\PY{p}{/}\PY{n+nt}{head}\PY{p}{\PYZgt{}}
\PY{p}{\PYZlt{}}\PY{n+nt}{body}\PY{p}{\PYZgt{}}
\end{Verbatim}

\end{framed}
\subsection{Odstavce}
Riadok, ktorý obsahuje iba medzery transformujte na riadok
\codefile{p.html}
\subsection{Veľký nadpis}
Ak riadok začína znakmi {\tt \#\textvisiblespace}, transformujte ho podľa
pravidla
\begin{center}
{\tt \#\textvisiblespace}{\em text}$\quad\to\quad${\tt <h1>}{\em text}{\tt </h1>}
\end{center}
Napríklad riadok
\codefile{h1.md}
sa transformuje na
\codefile{h1.html}
\subsection{Menší nadpis}
Ak riadok začína znakmi {\tt \#\#\textvisiblespace}, transformujte ho podľa
pravidla
\begin{center}
{\tt \#\#\textvisiblespace}{\em text}$\quad\to\quad${\tt <h2>}{\em text}{\tt </h2>}
\end{center}
Napríklad riadok
\codefile{h2.md}
sa transformuje na
\codefile{h2.html}
\subsection{Zvýrazňovanie}
Kdekoľvek na riadku sa nachádza reťazec typu {\tt \_\_}{\em text}{\tt \_\_},
\footnote{Znak \_ je podčiarkovník.}
nahraďte ho podľa pravidla
\begin{center}
{\tt \_\_}{\em text}{\tt \_\_}$\quad\to\quad${\tt <strong>}{\em text}{\tt </strong>}
\end{center}
Kdekoľvek na riadku sa nachádza reťazec typu {\tt \_}{\em text}{\tt \_},
nahraďte ho podľa pravidla
\begin{center}
{\tt \_}{\em text}{\tt \_}$\quad\to\quad${\tt <em>}{\em text}{\tt </em>}
\end{center}
Pritom {\em text} nesmie obsahovať podčiarkovníky.
Napríklad riadky
\codefile{emphasis.md}
sa transformujú na riadky
\codefilesmall{emphasis.html}
Dávajte si pozor na to, že riadok môže obsahovať viacero sekvencií horeuvedeného
typu.
\subsection{Pätička}
Na konci skript vypíše konštantný text
\begin{framed}
\begin{Verbatim}[commandchars=\\\{\}]
\PY{p}{\PYZlt{}}\PY{p}{/}\PY{n+nt}{body}\PY{p}{\PYZgt{}}
\PY{p}{\PYZlt{}}\PY{p}{/}\PY{n+nt}{html}\PY{p}{\PYZgt{}}
\end{Verbatim}

\end{framed}
\section{Pomôcky a návody}
Napíšem, ako som zadanie implementoval ja.

\begin{itemize}
\item
Použil som idióm {\tt while} -- {\tt read}
\codefile{whileread.sh}
Vo vnútri cyklu som potom mal k dispozícii postupne jednotlivé riadky {\em stdin} 
v premennej {\tt LINE}.
\item Testovanie, či obsah premennej {\tt LINE} sedí s regulárnym výrazom
{\tt regex} som robil podľa vzoru
\codefile{ifechogrep.sh}
Tento idióm funguje takto:
\begin{enumerate}
\item Konštrukt {\tt if {\em príkaz}} testuje, či exit status {\em príkaz} je rovný 0.
\item Exit status rúrovej sekvencie je exit status posledného príkazu v rúrovej sekvencii.
\item Príkaz {\tt grep 'regex'} má exit status 0 práve vtedy, keď nájde ten {\em
regex} na svojom vstupe
\item sekvencia ~~{\tt > /dev/null} slúži na zrušenie normálneho výstupu príkazu 
{\tt grep} -- chceme iba testovať, nechceme aby niečo vypisoval.
\end{enumerate}
\item Používal som idióm 
\codefile{transformline.sh}
pre transformáciu obsahu premennej {\tt LINE} pomocou {\tt sed}
\item Pripomínam, že vo {\tt while} je možné používať {\tt continue}.
\item V {\tt sed} som používal {\tt @} miesto {\tt /} na ohraničenie regulárneho výrazu,
pretože HTML obsahuje {\tt /}.
\item V {\tt sed} som používal spätnú referenciu cez {\tt \textbackslash 1} v časti {\tt replace}.
\item V {\tt sed} som používal flag {\tt g}, to jest {\tt 's@regex@replace@g'},
ak bolo treba.  
\end{itemize}

\section{Príklad vstupu a výstupu}
Je v tomto repozitári; {\tt example.md} je vstup a {\tt example.html} je výstup.
\end{document}
