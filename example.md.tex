\begin{Verbatim}[commandchars=\\\{\}]
\PY{g+gh}{\PYZsh{} O úpadku doby}
\PY{g+gu}{\PYZsh{}\PYZsh{} Karel Čapek}
Před jeskyní bylo ticho. Mužové odešli hned zrána mávajíce oštěpy směrem k
\PY{g+gs}{\PYZus{}\PYZus{}Blansku\PYZus{}\PYZus{}} nebo k \PY{g+gs}{\PYZus{}\PYZus{}Rájci\PYZus{}\PYZus{}}, kde bylo vystopováno stádo sobů; ženy zatím po lese
sbíraly bobule klikvy a jen časem bylo slyšet jejich ječivý pokřik a repetění;
děti se nejspíš čabraly dole v potoce – ostatně kdo by se těch harantů
dohlídal, holoty darebné a zvlčilé. A tož starý pračlověk \PY{g+ge}{\PYZus{}Janeček\PYZus{}} klímal v tom
vzácném tichu na mírném slunci říjnovém; po pravdě řečeno chrápal a v nose mu
hvízdalo, ale dělal, jako by nespal, nýbrž jako by střežil jeskyni kmene a
panoval nad ní, jakož se sluší na starého náčelníka.

\PY{g+ge}{\PYZus{}Janečková\PYZus{}} rozložila čerstvou kůži \PY{g+ge}{\PYZus{}medvědí\PYZus{}} a jala se ji oškrabovat ostrým
pazourkem. To se musí dělat důkladně, píď po pídi – a ne jako to dělá mladá,
napadlo starou Janečkovou; to famfárum to jen tak halabala odrbe a už zas běží
muchlat a muckat se s dětmi – takové kůže, myslí si stará Janečková, nic
nevydrží, kdepak, ztyří nebo se zapaří; ale já se jí do ničeho plést nebudu,
myslí si paní Janečková, když jí to neřekne syn – Jen co je pravda, šetřit
mladá neumí. A tady je ta kožišina propíchnuta, zrovna uprostřed hřbetu! Lidi
drazí, trne stará paní, kterýpak nekola toho medvěda píchal do zad? Vždyť se
tím zkazí celá kůže! To by můj jakživ neudělal, říká si stařena roztrpčeně, ten
se vždycky strefoval do krku – 

Stránka predmetu OS \PYZlt{}https://www.math.sk/wiki/OperacneSystemy\PYZgt{} je hento.

Dve URL na riadku \PYZlt{}https://www.google.com/\PYZgt{} aj toto \PYZlt{}https://www.stuba.sk\PYZgt{} je URL.

Zoznam nasleduje
 \PY{k}{\PYZhy{}} Zoznam
 \PY{k}{\PYZhy{}} nejakých
 \PY{k}{\PYZhy{}} vecí
a toto je nejaký iný text.
\end{Verbatim}
